\chapter {Axioms of Set Theory}


\section{}
Admit.


\section{}
If there is a set $X$ such that $P(X) \subset X$, let
\[ A = \{ x \in P(X) : x \notin x \}.\]
$A \subset P(X) $ hence $ A \subset X $, and then $ A \in P(X)$.
Considering whether or not $A \in A$ lead to contradiction.


\section{}
Let $ A = \{ x\in X : x \subset X \}$ .

$ \emptyset \in A $.

If $ x \in A $, namely $ x \in X $ and $ x \subset A $. For $ x^+ = x \bigcup \{x\} $,
$ x^+ \in X $ since $X$ is inductive. If some $ a \in x^+ $, either $ a \in x $ or
$ a = x $. In the first case, $ a \in X $ since $ x \subset X $; in the second case,
$ a = x \in X $. So $ x^+ \subset X $, hence $A$ is inductive.


\section{}
Let $ A = \{ x \in X : x \text{ is transitive} \} $ .

$ \emptyset \in A $.

Suppose $ x \in A $, namely $ x \in X $ and $ \forall{a}( a\in x \rightarrow a \subset x ) $,
and $ x^+ = x \bigcup \{x\} \in X $. For any $ a \in x^+ $, if $ a = x $ then $ a \subset x^+ $;
if $ a \in x $, then $ a \subset x $ since $x$ is transitive, hence $ a \subset x^+ $.


\section{}
Let $ A = \{ x \in X : x \text{ is transitive and } x \notin x \} $ .

$ \emptyset \in A $.

Suppose $ x \in A $, and $ x^+ = x \bigcup \{x\} \in X $. If $ x^+ \in x^+ $, then either 
$ x^+ \in x $ or $ x^+ = x $ according to its definition. The last case is impossible because 
it says $ x \in x $, which contradict with the hypothesis. In the first case, $ x^+ \subset x $
since $x$ is transitive, and because $ x \in x^+ $ so $ x \in x $, which also lead to contradiction.


\section{}
Let $ A = \{ x \in X : x \text{ is transitive and every nonempty } z \in x 
\text{ has an $\subset$-minimal element} \}  $ .

$ \emptyset \in A $.

Suppose $ x \in A $, and $ x^+ = x \bigcup \{x\} \in X $. Let $z$ be some nonempty subset of $x^+$.
If $ x \notin z $, then $z$ is also a subset of $x$, and have a minimal element. Otherwise, 
let $t$ be the minimal element of $ z - \{x\}\ \subset x $, we show that $t$ is also the minimal
element of $z$ by proof $x \notin t$. If $ x \in t $, we have $ x \in x $ (because $x$ is 
transitive so $ t \subset x $ ), which is impossible due to last exercise.


\section{}
Abort.


\section{}
Admit.


\section{}
$A$ is inductive, so $ N \subset A $ according to the definition of $N$. 
And $ A \subset N $, so $ A = N $.


\section{}
Abort.


\section{}
If $N$ is T-infinite, then $ N \subset P(N) $ has a $\subset$-maximal element $n$. 
But $ n^+ \in N $ and $ n \subset n^+ $, contradiction.


\section{}




