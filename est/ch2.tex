\chapter{Axioms and Operations}

\section{}
The set of integers divisible by 180.


\section{}
$ \bigcup\{\{a\}, \{b\}\} = \bigcup\{\{a,b\}\} = \{a,b\} $.


\section{}
Let $ a \in A $, for every $ x \in a $, we have $ x \in \bigcup A $.
Hence $ a \subseteq \bigcup A $.


\section{}
For every $ x \in \bigcup A $, there are some $a$ in $A$ that $ x \in a $.
Since $ A \subseteq B $, $a$ also belongs to $B$. Namely, $x$ belongs to
some member of $B$, so $ x \in \bigcup B $.


\section{}
For every $ x \in \bigcup A $, there are some $a$ in $A$ that $ x \in a $.
$ a \subseteq B $ according to assumption. So $ x \in B $.


\section{}
\begin{enumerate}
  \item
  If $ x \in \bigcup\mathscr{P}A $, then $ x \in a $ where $ a \in \mathscr{P}A $
  hence $ a \subseteq A $, so $ x \in A $. \\
  $ x \in A \quad \Rightarrow \quad  \{x\} \subseteq A  \quad \Rightarrow \quad
   \{x\} \in \mathscr{P}A  \quad \Rightarrow \quad   x \in \bigcup\mathscr{P}A $.

  \item
  $ a \in A  \quad \Rightarrow \quad  a \subseteq \bigcup A
    \quad \Rightarrow \quad  a \in \mathscr{P}\bigcup A $.
\end{enumerate}


\section{}
\begin{enumerate}
  \item
  \[ \begin{array}{rl}
  C \in \mathscr{P}A\bigcap\mathscr{P}B
  & \Leftrightarrow C \in \mathscr{P}A \ \&\ C \in \mathscr{P}B \\
  & \Leftrightarrow C \subseteq A \ \&\ C \subseteq B \\
  & \Leftrightarrow (\forall x \in C) x \in A \ \&\ x \in B  \\
  & \Leftrightarrow (\forall x \in C) x \in A\bigcap B \\
  & \Leftrightarrow C \subseteq A\bigcap B \\
  & \Leftrightarrow C \in \mathscr{P}(A\bigcap B)
  \end{array}\]

  \item
  \[ \begin{array}{rl}
  C \in \mathscr{P}A\bigcup\mathscr{P}B
  & \Rightarrow C \in \mathscr{P}A \ \mathnormal{or}\ C \in \mathscr{P}B \\
  & \Rightarrow C \subseteq A \ \mathnormal{or}\ C \subseteq B \\
  & \Rightarrow (\forall x \in C) x \in A \ \mathnormal{or}\ x \in B  \\
  & \Rightarrow (\forall x \in C) x \in A\bigcup B \\
  & \Rightarrow C \subseteq A\bigcup B \\
  & \Rightarrow C \in \mathscr{P}(A\bigcup B)
  \end{array}\]
\end{enumerate}


\section{}
For every set $A$, we can construct a singleton $ \{A\} $(that is,
$\{A,A\}$, according to Pairing Axiom),so if let $S$ be a set of all
singleton, then $ \bigcup S $ is a set of all sets.


\section{}
$ a = \{x, y\} $ and $ B = \{\{x, y\}\} $, where $x$ and $y$ are distinct.


\section{}
\[ \begin{array}{rl}
  a \in B
  & \Rightarrow (\forall x \in \mathscr{P}a) x \subseteq a \\
  & \Rightarrow (\forall x \in \mathscr{P}a)(\forall y \in x) y \in a \\
  & \Rightarrow (\forall x \in \mathscr{P}a)(\forall y \in x) y \in \bigcup B \\
  & \Rightarrow (\forall x \in \mathscr{P}a) x \subseteq \bigcup B \\
  & \Rightarrow (\forall x \in \mathscr{P}a) x \in \mathscr{P}\bigcup B \\
  & \Rightarrow \mathscr{P}a \subseteq \mathscr{P}\bigcup B \\
  & \Rightarrow \mathscr{P}a \in \mathscr{PP}\bigcup B \\
\end{array}\]


\section{}
\[ \begin{array}{rl}
                  & x \in (A\bigcap B)\bigcup (A-B) \\
  \Leftrightarrow & x \in (A\bigcap B) \ \mathnormal{or}\ x \in (A-B) \\
  \Leftrightarrow & (x\in A\ \&\ x\in B)\mathnormal{or}(x\in A\ \&\ x\notin B) \\
  \Leftrightarrow & (x\in A)\mathnormal{or}(x\in B\ \mathnormal{or}\ x\notin B) \\
  \Leftrightarrow & x\in A
\end{array}\]

\[ \begin{array}{rl}
                  & A \bigcup B \\
  \Leftrightarrow & A \bigcup ((B\bigcap A)\bigcup(B-A)) \\
  \Leftrightarrow & A \bigcup (B\bigcap A) \bigcup(B-A) \\
  \Leftrightarrow & A \bigcup (B\bigcap A) \\
\end{array}\]


\section{}
Admit.


\section{}
If $ x \in C-B $, then $ x\in C\ \&\ x\notin B $. So $x$ also doesn't belong to
$A$(because if so, then $ x\in B $ according to $ A\subseteq B $, which led to 
contradiction), that is, $ x\in C\ \&\ x\notin A $, or $ x \in C-A$.


\section{}
Let $ A = \{1, 2\} $, $ B = \{4\} $ and $ C = \{2, 3\} $, then $ (A-(B-C)) = \{1, 2\} $
while $ (A-B)-C = \{1\} $.


\section{}
Admit.


\section{}
\[ \begin{array}{rl}
    & [(A\bigcup B\bigcup C)\bigcap (A\bigcup B)]-[(A\bigcup(B-C))\bigcap A] \\
  = & (A\bigcup B)-A \\
  = & B-A
\end{array}\]


\section{}
Admit.

\section{}
Admit.

\section{}
No. Yes.
