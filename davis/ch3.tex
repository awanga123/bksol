\chapter{Primitive Recursive Functions}

\section{Composition}
No exercise.


\section{Recursion}
No exercise.


\section{PRC Classes}

\subsection{}
\[ h_{1}(x, y, z) = g_{1}(u^{3}_{3}(x, y, z), u^{3}_{2}(x, y, z), u^{3}_{1}(x, y, z)) \]
\[ h_{2}(x) = g_{1}(u^{1}_{1}(x), u^{1}_{1}(x), u^{1}_{1}(x)) \]
\[
\begin{split}
h_{3}(w, x, y, z) = h_{1}( & g_{3}(u^{4}_{1}(w, x, y, z), u^{4}_{3}(w, x, y, z)), \\
                           & u^{4}_{4}(w, x, y, z), \\
                           & g_{4}( s(s(n(u^{4}_{1}(w, x, y, z)))), g_{4}(u^{4}_{3}(w, x, y, z), u^{4}_{4}(w, x, y, z))))
\end{split}
\]


\subsection{}
Three initial functions are all total; functions obtained from total functions
by either composition or recursion also total.

\subsection{}
Initial function $ u^{n+1}_{i}(\dots) $ does not belong to $ \mathscr{C} $.

\subsection{}
Admit.

\subsection{}
Admit.



\section{Some Primitive Recursive Functions}

\subsection{}
Admit.

\subsection{}
\begin{align*}
  f(0)   & = k \\
  f(t+1) & = f(t)
\end{align*}


\subsection{}
Admit.

\subsection{}
Admit.


\subsection{}
\begin{align*}
  \iota_{f}(0, x)   & = u^{1}_{1}(x) \\
  \iota_{f}(t+1, x) & = f(\iota_{f}(t, x))
\end{align*}


\subsection{}
\begin{enumerate}
  \item
  \begin{align*}
    E(0)   & = 0 \\
    E(t+1) & = \alpha(E(t))
  \end{align*}

  \item
  \begin{align*}
    H(0)   & = 0 \\
    H(t+1) & = H(t) + E(t)
  \end{align*}
\end{enumerate}


\subsection{}
\[\ f(x) = h(x, x) \]
where $h(x,y)$ is primitive recursive :
\begin{align*}
  h(x, 0)   & = 1 \\
  h(x, t+1) & = {h(x, t)}^{x} \\
\end{align*}


\subsection{}
$ f(x+1) < x+1 $ for all $x$, particularly, $ f(1) < 1 $ then must have
 $ f(1) = 0 $. So $ f'(x) = \min _{t\le x} f^{t}(x) = 0 $ is defined on 
all $ x>0 $, and is computable:
\begin{center}
\begin{tabular}{ll}
  $[A]$ & $ Z \gets f^{Y}(x) $ \\
        & IF $Z = 0$ GOTO $E$ \\
        & $ Y \gets Y + 1 $ \\
        & GOTO $A$
\end{tabular}
\end{center}
According to Exercise 5, $ g^{n}(x) = \iota_{g}(n, x) $ is computable.
Then $h$ can be defined as:
\begin{align*}
  h(0)   & = k \\
  h(t+1) & = g^{f'(t+1)}(k) \\
\end{align*}


\subsection{}
\[ f(n, x) = g^{2^{n}}(x) = \iota_{g}(2^{n}, x) \]

\subsection{}
Admit.

\subsection{}
\begin{enumerate}
  \item Admit.
  
  \item 
  $ g(x_{1}, x_{2}) \le \max\{x_{1}, x_{2}\} + k $ for some $k$ according to (a).
  When $ t = 0 $, 
  \[ h(0) = c \le 0 \cdot k + c .\]
  Suppose the result is known for $ t $:
  \[ h(t) \le t \cdot k + c .\]
  Then for $ t+1 $:
  \begin{align*}
    h(t+1) & = g(t, h(t)) \\
           & \le \max\{t, h(t)\} + k \\
           & \le \max\{t, t \cdot k + c\} + k \\
  \end{align*}
  Because $ t \le t \cdot k + c $ (since $k>0$), 
  \[ h(t+1) \le (t+1)k+c \]

  \item Similar to (b).
  \item Admit.
  \item Admit.
\end{enumerate}



\section{Primitive Recursive Predicates}


