\chapter{A Universal Program}


\section{Coding Programs by Numbers}

\subsection{}
Admit.

\subsection{}
\begin{align*}
  575+1 & = 2^6 \cdot 3^2 = [6,2] \\
  6     & = <0,3> = <0,<2,0>> \\
  2     & = <0,1> = <0,<1,0>> \\
\end{align*}
Thus, program $\mathscr{P}$ is:
\begin{align*}
  Y \gets Y - 1 \\
  Y \gets Y + 1
\end{align*}



\section{The Halting Problem}

\subsection{}
Admit.

\subsection{}
If $ \mathrm{\overline{HALT}}(x,y) $ is computable, then so is
$ \mathrm{HALT}(x,y)$ :
\begin{align*}
  & \mathrm{IF}\ \mathrm{\overline{HALT}}(X_1,X_2)\neq 0  \ \mathrm{GOTO}\ E \\
  & Y \gets Y + 1
\end{align*}


\subsection{}
\[ \mathrm{HALT}(x,y) = \mathrm{HALT}^1(<x,y>) \]


\subsection{}
$ \mathrm{HALT}(x,y) $ is total and $ \mathrm{HALT}(x,y) \le 1 $ for all $ x,y $.
But it is not computable.


\subsection{}
Suppose $\#(\mathscr{P}) = n $.
Let program $\mathscr{Q}$ be $\mathscr{P}$ followed instruction:
\[ [E] \text{ IF } Y \neq 0 \text{ GOTO } E \]
then
\[ \psi_{\mathscr{Q}}(x) = \sim \mathrm{HALT}(x,x) \]
and
\[ \#(\mathscr{Q}) = n\cdot P_{Lt(n)+1}^{<5,<7,0>>} .\]
So
\[ \psi_{\mathscr{P}}(\#(\mathscr{Q}))
   \Leftrightarrow \mathrm{HALT}(\#(\mathscr{Q}, \#(\mathscr{Q})
   \Leftrightarrow \sim \mathrm{HALT}(\#(\mathscr{Q}, \#(\mathscr{Q}) .\]


\subsection{}
Goldbach's conjecture is either true or false, and has nothing to do with
the input $x$. If it's true, then $ f(x) = x $; otherwise, $ f(x) = 0 $.
In both cases, $f(x)$ is primitive recursive.



\section{Universality}

\subsection{}
Admit.

\subsection{}
\begin{enumerate}
  \item
  \begin{align*}
    & Z \gets \Phi(X,X) \\
    & Y \gets Y + 1
  \end{align*}

  \item
  \begin{align*}
    & \text{IF } X = a_{1} \text{ GOTO } E \\
    & \text{IF } X = a_{2} \text{ GOTO } E \\
    & \cdots \\
    & Z \gets \Phi(X,X) \\
    & Y \gets Y + 1
  \end{align*}

  \item
  Suppose program $\mathscr{P}$ begin with following instruction:
  \[ [A] \text{ IF } X \neq 0 \text{ GOTO } A \]
  then $\mathscr{P}$ is undefined for all $ x \neq 0 $. Since
  $ \#(\mathscr{P}) > 0 $, so $ \Phi(\#(\mathscr{P}), \#(\mathscr{P}))\uparrow $
  for such $\mathscr{P}$. Take
  \[ B = \{b \ |\ (b)_1 = <1,<3,1>> \} ,\]
  Then $ H_3(x) $ is partially computable:
  \begin{align*}
    & \text{IF } (X)_1 = <1,<3,1>> \text{ GOTO } E \\
    & \cdots \\
    & Z \gets \Phi(X,X) \\
    & Y \gets Y + 1
  \end{align*}

  \item
  \[ C = \{ x\ |\ \Phi(x,x)\uparrow \} \]
  Then
  \[ \begin{split}
    H_4(x) & \Leftrightarrow \Phi(x,x)\downarrow \\
           & = \mathrm{HALT}(x,x)
  \end{split} \]
  which is not computable.
\end{enumerate}


\subsection{}
If $\mathscr{P}$ computes $\Phi^{(1)}(x,y) $, then
\[\begin{split}
  H_{\mathscr{P}}(x,y) & \Leftrightarrow \Phi^{(1)}(x,y) \text{ halts on inputs } x,y \\
                       & = \mathrm{HALT}(x,y)
\end{split}\]


\subsection{}
\[ f(x_1,x_2,\dots,x_n) =
   (r(\mathrm{SNAP}^{(n)}(x_1,x_2,\dots,x_n,\#(\mathscr{P}),g(x_1,x_2,\dots,x_n))))_1 \]


\subsection{}
Abort.

\subsection{}
Abort.



\section{Recursively Enumerable Sets}

\subsection{}
Admit.


\subsection{}
\[ K_0 = \{ x\in N\ |\ \Phi(l(x), r(x))\downarrow \} \]


\subsection{}
\begin{enumerate}
  \item
  \[ \mathrm{gr}(f) = \{ x\in N \ |\ f((x)_1, (x)_2, \dots, (x)_n) = (x)_{n+1} \} \]

  \item
  If $ \mathrm{gr}(f) $ is recursive, then
  $ \mathrm{gr}(f) = \{ x\in N \ |\ g(x) \} $
  where $ g(x) $ is a recursive predicate.
  Then the following program computes $ f(x) $:
  \begin{align*}
    [A]\ & \text{IF } g([X_1,X_2,\dots,X_n,Y]) \text{ GOTO } E \\
         & Y \gets Y + 1 \\
         & \text{GOTO } A
  \end{align*}

  \item 
  \[ f(x_1,x_2) = 
     \begin{cases}
       x_2 - x_1 & \text{if } x_2 \ge x_1 \\
       \uparrow  & \text{otherwise}
     \end{cases} \]
  \[ g(x) \Leftrightarrow Lt(x) = 3\ \&\ (x)_1 + (x)_3 = (x)_2 \]
  $ g(x) $ is computable but $ f(x_1,x_2) $ is not, because it's not total.
\end{enumerate}


\subsection{}
Let $ g(x) \Leftrightarrow (\exists t\in N)f(t)=x $, then it's computed by
following program:
\begin{align*}
  [A]\ & \text{IF } f(Z)=X \text{ GOTO } B \\
       & \text{IF } f(Z)>X \text{ GOTO } E \\
       & Z \gets Z + 1 \\
       & \text{GOTO } A \\
  [B]\ & Y \gets Y + 1 \\
\end{align*}
Since $ f(x)\ge x $, this program always halts when or before $Z$ growing
to $X$. So, it's both partially computable and total.


\subsection{}
If $A$ is r.e, then $ A = \{ f(x)\ |\ x\in N \} $ for some computable function
$f(x)$. Since $A$ is infinite, namely range of $f(x)$ is infinite, so
\[ (\forall x\in N)(\exists y>x)f(y)>f(x) \]
Then $ next(x) = \min_{y\ge x} f(y)>f(x) $ is total thus computable:
\begin{align*}
       & Z \gets X \\
  [A]\ & \text{IF } f(Z)\ge f(X) \text{ GOTO } E \\
       & Z \gets Z + 1 \\
       & \text{GOTO } A \\
\end{align*}
so as $g(x)$:
\begin{align*}
  g(0)   & = 0 \\
  g(t+1) & = next(g(t)) .\\
\end{align*}
Then $ h(x) = f(g(x)) $ is a strictly increasing computable function and 
range of $h(x)$ is a subset of $f(x)$ . So r.e. set
\[ B = \{ h(x)\ |\ x\in N \} \]
is an infinite subset of $A$, and, by last exercise, is recursive.


\subsection{}
Abort.


\subsection{}
\begin{enumerate}
  \item 
  $K$ and $\overline{K}$ are two sets, and $ K \bigcup \overline{K} = N $. $N$ is evidently
  r.e., but $ \overline{K} $ is not.
  
  \item 
  $ \overline{K} \subseteq N $.S
\end{enumerate}

\subsection{}
Suppose there is computable function $f(x)$ such that:
\[ f(x) =
     \begin{cases}
       \Phi(x,x)+1 & \text{if } \Phi(x,x)\downarrow \\
       0           & \text{otherwise}
     \end{cases} \]
Then
\[ \alpha(\alpha(f(x))) \Leftrightarrow \mathrm{HALT}(x,x) \]
which is not computable.


\subsection{}
\begin{enumerate}
  \item 
  Suppose $p$ is the number of program computes $g(x)$, and $q$ is
  that for $h(x)$.
  \begin{align*}
    [A]\quad & \text{IF } STP^{(1)}(X,p,Z) \text{ GOTO } B \\
         & \text{IF } STP^{(1)}(X,q,Z) \text{ GOTO } C \\
         & Z \gets Z + 1 \\
         & \text{GOTO } A \\
    [B]\quad & Y \gets g(X) \\
         & \text{GOTO } E \\
    [C]\quad & Y \gets h(X) \\
         & \text{GOTO } E \\
  \end{align*}
  
  \item 
  No. According to Exercise 3.2, it's possible to compute $g(x)$ such that:
  \[ g(x) = 
     \begin{cases}
       1        & \Phi(x,x) \downarrow \\
       \uparrow & \text{otherwise}
     \end{cases} \]
  Take $ h(x) = 0 $ for every $x$, if
  \[ f(x) =
     \begin{cases}
       g(x)    & \text{if } g(x) \downarrow \\
       h(x)     & \text{if } g(x) \uparrow \text{ and } h(x) \downarrow \\
       \uparrow & \text{otherwise}
     \end{cases} \]
  then
  \[ f(x) = 
     \begin{cases}
       1 & \text{if } \Phi(x,x) \downarrow \\
       0 & \text{otherwise}
     \end{cases}
     \Leftrightarrow \mathrm{HALT}(x,x) \]
\end{enumerate}


\subsection{}
\begin{enumerate}
  \item 
  Let $f(x)$ be partially computable function computed by following program:
  \begin{align*}
    [A]\quad & \text{IF } P(Z,X) \text{ GOTO } E \\
             & Z \gets Z + 1 \\
             & \text{GOTO } A 
  \end{align*}
  Then
  \[ A = \{ x\in N\ |\ f(x)\downarrow \} \]
  So $A$ is r.e.
  
  \item 
  \begin{align*}
    [A]\quad & \text{IF } Q((Z)_1,(Z)_2,\dots,(Z)_n,X) \text{ GOTO } E \\
             & Z \gets Z + 1 \\
             & \text{GOTO } A
  \end{align*}
\end{enumerate}


\subsection{}
\begin{align*}
  \overline{K} & = \{ x\in N\ |\ \Phi(x,x)\uparrow \} \\
               & = \{ x\in N\ |\ (\forall t)\sim STP^{(1)}(x,x,t) \}
\end{align*}
$\sim STP^{(1)}(x,x,t)$ is computable but $\overline{K}$ is not r.e.
